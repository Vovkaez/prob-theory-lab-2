\documentclass{article}
\usepackage{graphicx} % Required for inserting images
\usepackage[utf8]{inputenc}
\usepackage[english,russian]{babel}
\usepackage[left=2cm,right=2cm,top=2cm,bottom=2cm,bindingoffset=0cm]{geometry}
\usepackage{mathabx}
\usepackage{amsmath}


\title{Лабораторная работа №2}
\author{Владимир Медведев, группа M3238}
\date{}

\begin{document}

\maketitle

\section*{Задача №1, вариант 3}

Заметим, что $X$ - дискретная случайная величина. Тогда её можно представить только как сумму дискретных. Дадим название с.в., на которые мы раскладываем, пусть X = Y + Z, Y и Z независимы, одинаково распределены и дискретны. Какие значения могут принимать Y и Z с ненулевой вероятностью? Обозначим их множество как $D$. Рассмотрим случай Y = Z \\

\[ x \in D \iff 2x \in \{-1, 0, 1 \} \iff x \in \{-0.5, 0, 0.5\} \] \\

Выразим $p_1, p_2, p_3$ с помощью распределения $Y$ и $Z$. Обозначим $q_1 = P(Y = -0.5), q_2 = P(Y = 0), q_3 = P(Y = 0.5)$ \\
\[ p_1 = q_1^2, p_2 = q_2^2 + 2 q_1 q_3, p_3 = q_3^2\]
При этом все другие исходы невозможны, то есть
\[ q_1q_2=q_2q_3 = 0 \]
Тогда носитель Y и Z либо состоит из 1 числа, либо равен $ \{ -0.5, 0.5 \} $. В терминах $p_1, p_2, p_3$ эти условия равносильны
\[ p_2 = 1 \lor \sqrt{p_1} + \sqrt{p_3} = 1 \]
Этих условий будет достаточно, так как при их выполнении можно выразить $q$.
\[ p_2 = 1 \implies q_2 = 1, q_1 = q_3 = 0 \]
\[ \sqrt{p_1} + \sqrt{p_3} = 1 \implies q_1 = \sqrt{p_1}, q_3 = \sqrt{p_3}, q_2 = 0 \]

\section*{Задача №2, вариант 2}
Покажем, что $(X, Y)$ - гауссовский вектор с нулевыми матожиданиями и дисперсией $\sigma ^ 2 * E$. Для этого найдём совместную плотность $p_{X, Y}$ с помощью формулы
\[ p_{X, Y}(x, y) = p_{R, \big \theta}(r(x, y), \theta(x, y)) \left| det \begin{pmatrix} \nabla r(x, y) \\ \nabla \theta(x, y) \end{pmatrix} \right| \]
Поскольку $R$ и $\big \theta$ независимы

\[ p_{R, \big \theta}(r, \theta) = p_R(r) p_{\big \theta}(\theta) = \frac{x}{2 \pi \sigma ^ 2}exp \left(-\frac{x^2}{2 \sigma ^ 2}\right) \]

\[ r(x, y) = \sqrt{x ^ 2 + y ^ 2} \]

\[ \theta(x, y) = \arctan{\frac{Y}{X}} \]

\[ \left| det \begin{pmatrix} \nabla r(x, y) \\ \nabla \theta(x, y) \end{pmatrix} \right| = \left| det \begin{pmatrix} \frac{x}{\sqrt{x^2 + y^2}} & \frac{y}{\sqrt{x^2 + y^2}} \\ -\frac{y}{x^2} \frac{1}{1 + \frac{y^2}{x^2}} & \frac{1}{x} \frac{1}{1 + \frac{y^2}{x^2}} \end{pmatrix} \right| = \left| det \begin{pmatrix} \frac{x}{\sqrt{x^2 + y^2}} & \frac{y}{\sqrt{x^2 + y^2}} \\ -\frac{y}{x^2 + y^2} & \frac{x}{x^2 + y^2} \end{pmatrix} \right| = \left| \frac{x^2 + y^2}{(x^2 + y^2)^{\frac{3}{2}}}  \right| =  \frac{1}{\sqrt{x^2 + y^2}}\]

\[ p_{X, Y}(x, y) = \frac{\sqrt{x^2 + y^2}}{2 \pi \sigma ^ 2} exp \left( -\frac{x^2 + y^2}{2 \sigma^2} \right) \frac{1}{\sqrt{x^2 + y^2}} = \frac{1}{2 \pi \sigma ^ 2} exp \left( -\frac{x^2 + y^2}{2 \sigma^2} \right) \]

Это и есть совместная плотность распределения соответствующего гауссовского вектора. Тогда $X$ и $Y$ являются нормально распределёнными с матожиданием $0$ и дисперсией $\sigma ^ 2$ как компоненты гауссовского вектора и независимыми, поскольку соответсвующие ковариации в матрице равны нулю.
    
\end{document}
